\documentclass[a4paper, lualatex, ja=standard]{bxjsarticle}

\usepackage{geometry}
\geometry{
  left=2.5cm, right=2.5cm, top=3cm, bottom=3cm
}

\usepackage{amsmath, amsthm, amssymb}
\theoremstyle{theorem}
\newtheorem{thm}{Theorem}[section]
\newtheorem*{thm*}{Theorem}
\newtheorem{prop}[thm]{Proposition}
\newtheorem*{prop*}{Proposition}

\theoremstyle{definition}
\newtheorem{rem}[thm]{Remark}
\newtheorem{expl}[thm]{Example}
\newtheorem{excs}{Exercise}[section]

% \renewcommand{\proofname}{\textbf{証明}}
% \newcommand{\F}{\mathcal{F}}
% \newcommand{\E}{\mathbb{E}}
\newcommand{\diff}{\mathrm{d}}

% \renewcommand{\labelenumi}{(\roman{enumi})}
\renewcommand{\theequation}{\thesection.\arabic{equation}}
% \makeatletter
% \@addtoreset{equation}{subsection}
% \makeatother

\setlength{\parindent}{0cm}

\usepackage{hyperref}




\title{Stochastic Interest Rates 第1回}
\author{朝倉響}
\date{2024年8月7日}

\begin{document}
\maketitle

\section*{概要}
\begin{itemize}
  \item tmp
\end{itemize}

\section{固定利付商品(債券):Fixed-income instruments}
\begin{itemize}
  \item 以下で議論するのはコール市場の金利(interbank rates)
  \item 最も重要なのはLIBOR (London Interbank Offered Rate)
  \begin{itemize}
    \item ロンドン市場の主要銀行が提示する無担保短期金利の平均
    \item 不正利用により廃止された
  \end{itemize}
\end{itemize}
\subsection{Interest rates and bonds}
\begin{itemize}
  \item 満期$T$の\emph{ゼロクーポン債}(\emph{zero-coupon bond}),\emph{割引債}(\emph{discount bond})
\end{itemize}
\begin{rem}
\end{rem}

\begin{rem}
\end{rem}

\begin{excs}
\end{excs}

\begin{excs}
\end{excs}

\begin{prop}
\end{prop}

\subsection{Forward rate agreements}
\begin{prop}
\end{prop}

\begin{excs}
\end{excs}
\begin{excs}
\end{excs}

\subsection{Forward interest rates and forward bond price}
\begin{excs}
\end{excs}

\begin{rem}
\end{rem}

\begin{excs}
\end{excs}

\begin{rem}
\end{rem}

\begin{expl}
\end{expl}

\subsection{Money market account}

\subsection{Coupon-bearing bonds}
\begin{excs}
\end{excs}



\end{document}