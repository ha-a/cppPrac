\documentclass[a4paper, lualatex, ja=standard]{bxjsarticle}

\usepackage{geometry}
\geometry{
  left=2.5cm, right=2.5cm, top=3cm, bottom=3cm
}

\usepackage{amsmath, amsthm, amssymb, mathtools}
\usepackage{xcolor}
\theoremstyle{definition}
\newtheorem{thm}{定理}[subsection]
\newtheorem*{thm*}{定理}
\newtheorem{dfn}[thm]{定義}
\newtheorem*{dfn*}{定義}
\newtheorem{lem}[thm]{補題}
\newtheorem*{lem*}{補題}
\newtheorem{expl}[thm]{例題}
\newtheorem{note}[thm]{注意}
\newtheorem*{note*}{注意}
\newtheorem*{prop*}{命題}
\newtheorem*{assum*}{仮定}
\renewcommand{\proofname}{\textbf{証明}}
\newcommand{\F}{\mathcal{F}}
\renewcommand{\P}{\mathbb{P}}
\newcommand{\E}{\mathbb{E}}
\newcommand{\PP}{\widetilde{\mathbb{P}}}
\newcommand{\EE}{\widetilde{\mathbb{E}}}
\newcommand{\WW}{\widetilde{W}}
\newcommand{\diff}{\mathrm{d}}

\renewcommand{\labelenumi}{(\roman{enumi})}
\renewcommand{\theequation}{\thesubsection.\arabic{equation}}
\makeatletter
\@addtoreset{equation}{subsection}
\makeatother

\setlength{\parindent}{0cm}

\usepackage{hyperref}

\title{金融工学勉強会第5回}
\author{朝倉響}
\date{2024年7月26日}

\begin{document}
\maketitle

\setcounter{section}{2}
\section*{前回の復習と今回の概要}
\begin{itemize}
  \item 前回(2024年8月16日)
  \begin{itemize}
    \item 
  \end{itemize}
  \item 今回(2024年8月23日)
  \begin{itemize}
    \item 
  \end{itemize}
\end{itemize}

\setcounter{section}{5}
\setcounter{subsection}{3}
\subsection{資産価格評価の基本定理}
表記
\begin{itemize}
  \item $\P$:実確率測度
  \item $\F(t)$:ブラウン運動に関連するフィルトレーション
  \begin{itemize}
    \item 必ずしも「ブラウン運動から生成するフィルトレーション」とは限らない
  \end{itemize}
  \item $\F=\F(T)$
  \item $W(t)$:$(\Omega,\F,\P)$上$d$次元ブラウン運動
\end{itemize}
\subsubsection{ギルサノフの定理とマルチンゲールの表現定理}
\begin{thm}[多次元ギルサノフの定理]
  $\Theta(t)$:$d$次元適合過程
  \begin{align}
    Z(t) &\coloneqq \exp \left\{-\int_0^t \Theta(u) \cdot \diff W(u)-\frac{1}{2} \int_0^t\|\Theta(u)\|^2 \,\diff u\right\} \\
    Z &\coloneqq Z(T) \notag\\
    \widetilde{W}(t) &\coloneqq W(t)+\int_0^t \Theta(u) \,\diff u\\
    &\text{ただし}\E \int_0^T\|\Theta(u)\|^2 Z^2(u) \,\diff u<\infty
  \end{align}
  とすると,$\E Z=1$でかつ,確率測度$\PP$
  \begin{equation*}
    \PP(A)\coloneqq \int_A Z(\omega) \,\diff \P(\omega) ,\quad A \in \F
  \end{equation*}
  に対し,$\WW(t)$は$\PP$の下で$d$次元ブラウン運動.
\end{thm}
\begin{proof}
  $\WW(t)$は時刻$0$で値$0$を取り,連続な経路を持つマルチンゲール(マルチンゲールである証明の流れは$1$次元の場合と同様である\footnote{$Z(t)$がラドン--ニコディム過程であること,$\WW(t)Z(t)$がマルチンゲールなことを示し,補題5.2.2を使う}.ただし,多次元の計算が含まれることに注意).
  2次変分を考える.
  \begin{align*}
    \diff\WW_i(t)\diff\WW_j(t) &= (\diff W_i(t)+\Theta_i(t)\diff t)(\diff W_j(t)+\Theta_j(t)\diff t)\\
    &= \diff W_i(t)\diff W_j(t) = \begin{cases}
      \diff t & (i=j)\\
      0 & (i\neq j)
    \end{cases}\quad\text{($W(t)$は$d$次元ブラウン運動)}
  \end{align*}
  結果,レヴィの定理より$\WW(t)$は$\PP$上$d$次元ブラウン運動.
\end{proof}
\begin{itemize}
  \item $\P$と$\PP$は同値\footnote{
    一般の測度変換は$\PP\ll\P$だが,いま$Z$はほとんど確実に正であり,$\PP(B)=0$なる$B$に対し
    \begin{equation*}
      \P(B) = \E\mathbb{I}_B = \widetilde{E}\left[\frac{\mathbb{I}_B}{Z}\right] = 0
    \end{equation*}
    より$\PP\gg\P$.
  }:実確率測度$\P$とリスク中立確率測度$\PP$で何が起こり得るかは一致.
  \item $\WW(t)$の各成分は$\P$のもとで独立でないこともあるが,$\PP$のもとでは独立.
\end{itemize}

\begin{thm}[多次元のマルチンゲール表現定理]
  仮定
  \begin{itemize}
    \item $\F(t)$:$d$次元ブラウン運動$W(t)$から生成されるフィルトレーション(より強い仮定)
    \item $M(t)$:$\F(t)$に関するマルチンゲール
  \end{itemize}
  このとき,$d$次元適合仮定$\Gamma(t)$が存在して
  \begin{equation}
    M(t) = M(0)+\int_0^t \Gamma(u) \cdot \diff W(u).
  \end{equation}
  さらに,定理5.4.1の仮定が成り立ち,$\widetilde{M}(t)$が$\PP$--マルチンゲールとすると,$\widetilde{\Gamma}(t)$が存在して
  \begin{equation}
    M(t) = M(0)+\int_0^t \widetilde{\Gamma}(u) \cdot \diff \WW(u).
  \end{equation}
\end{thm}
\begin{proof}
  \textcolor{red}{(証明は省略)}
\end{proof}

\subsubsection{多次元市場モデル}
$m$種類株式
\begin{align}
  \diff S_i(t) &= \alpha_i(t)S_i(t) \diff t+\sum_{j=1}^d \sigma_{ij}(t)S_j(t) \,\diff W_j(t),\quad i=1,\ldots,m\\
  &\left.\begin{aligned}
    &\text{期待収益率ベクトル: }(\alpha_i(t))_{i=1,\dots,m} \\
    &\text{ボラティリティ行列: }(\sigma_{ij}(t))_{i=1,\dots,m;j=1,\dots,d} 
  \end{aligned}\quad\right\}\text{適合過程}\notag
\end{align}
ここから調べたいこと
\begin{itemize}
  \item $S_i(t)$のボラティリティ
  \item $S_i$と$S_k$の相関
  \item 多次元市場モデルでのリスク中立測度が存在する必要十分条件
  \begin{itemize}
    \item リスク中立測度が存在すると何が嬉しいか
    \item このリスク中立測度は一意か
  \end{itemize}
\end{itemize}

\subsubsection*{$S_i$のボラティリティ}
以下を定義する
\begin{align}
  \sigma_i(t) &= \sqrt{\sum_{j=1}^d \sigma_{ij}^2(t)}\notag\\
  B_i(t) &= \sum_{j=1}^d \int_0^t\frac{\sigma_{ij}(t)}{\sigma_i(t)}\diff W_j(t)\\
  &\text{連続なマルチンゲール}\notag\\
  &\diff B_i(t)\diff B_i(t) = \sum_{j=1}^d\frac{\sigma_{ij}^2(t)}{\sigma_i^2(t)}\diff W_j(t)\diff t = \diff t\notag
\end{align}
このとき,$B_i(t)$は(1次元)レヴィの定理よりブラウン運動.
\begin{equation}
  \diff S_i(t) = \alpha_i(t)S_i(t) \,\diff t + \sigma_i(t)S_i(t)\,\diff B_i(t)
\end{equation}
$S_i(t)$のボラティリティは$\sigma_i(t)$.

\subsubsection*{$S_i$と$S_k$の相関}
\setcounter{equation}{12}

\begin{align}
  \diff S_i(t)\diff S_k(t) &= \sigma_i(t)\sigma_k(t)S_i(t)S_k(t)\,\diff B_i(t)\diff B_k(t) \notag\\
  &= \sigma_i(t)\sigma_k(t)S_i(t)S_k(t)\underbrace{\sum_{j=1}^d\frac{\sigma_{ij}(t)\sigma_{kj}(t)}{\sigma_i(t)\sigma_k(t)}}_{\eqqcolon\rho_{ij}(t)}\diff t \notag\\
  &= \rho_{ik}(t)\sigma_i(t)\sigma_k(t)S_i(t)S_k(t)\,\diff t
\end{align}
\begin{itemize}
  \item $\sigma_i(t)$:$S_i$の時刻$t$での相対変化の瞬間的標準偏差
  \item $\rho_{ik}(t)$:$S_i$と$S_k$の相対変化の瞬間的な相関
\end{itemize}

\subsubsection{リスク中立測度の存在}
\begin{dfn}[リスク中立]
  条件
  \begin{enumerate}
    \item $\PP$と$\P$が同値
    \item 割り引かれた株価$D(t)S_i(t)$が$\PP$の下でマルチンゲール($\forall i=1,\ldots,m$)
  \end{enumerate}
  を満たす確率測度$\PP$:リスク中立測度.
\end{dfn}
割り引かれた株価過程
\begin{itemize}
  \item $R(t)$:金利過程(適合過程)
  \item $D(t) = e^{-\int_0^t R(u)\,\diff u}$:割引過程
  \item $D(t)S_i(t)$:割り引かれた株価
  \setcounter{equation}{14}
  \begin{align}
    \diff(D(t)S_i(t)) &= D(t)\diff S_i(t)+S_i(t)\diff D(t)+\diff S_i(t)\diff D(t)\notag\\
    &= D(t)\left(\alpha_i(t)S_i(t)\,\diff t+\sigma_i(t)S_i(t)\,\diff B_i(t)\right)-S_i(t)R(t)D(t)\,\diff t\notag\\
    &= D(t)S_i(t)\left[(\alpha_i(t)-R(t))\,\diff t+\sigma_i(t)\,\diff B_i(t)\right]
  \end{align}
\end{itemize}


\end{document}