\documentclass[a4paper, lualatex, ja=standard]{bxjsarticle}

\makeatletter
\def\thesis#1{\def\@thesis{#1}}
\def\dept#1{\def\@dept{#1}}
\def\@maketitle{
  \begin{center}
    {\Large \@title \par}
  \end{center}
  \hfill{\normalsize \@date\par}
  \hfill{\normalsize \@dept}\hspace{0.5cm}{\large \@author\par}
  \vskip1em
}
\makeatother


\usepackage{geometry}
\geometry{
  left=2.5cm, right=2.5cm, top=3cm, bottom=3cm
}

\usepackage{amsmath, amsthm, amssymb}
\theoremstyle{theorem}
\newtheorem{thm}{Theorem}[section]
\newtheorem*{thm*}{Theorem}
\newtheorem{prop}[thm]{Proposition}
\newtheorem*{prop*}{Proposition}

\theoremstyle{definition}
\newtheorem{rem}[thm]{Remark}
\newtheorem{expl}[thm]{Example}
\newtheorem{excs}{Exercise}[section]

% \renewcommand{\proofname}{\textbf{証明}}
\newcommand{\F}{\mathcal{F}}
\newcommand{\E}{\mathbb{E}}
\newcommand{\I}{\mathbf{1}}
\newcommand{\FP}{\mathbf{FP}}
\newcommand{\BC}{\mathbf{BC}}
\newcommand{\BP}{\mathbf{BP}}
\newcommand{\diff}{\mathrm{d}}

% \renewcommand{\labelenumi}{(\roman{enumi})}
\renewcommand{\theequation}{\thesection.\arabic{equation}}
\newcommand{\relmiddle}[1]{\mathrel{}\middle#1\mathrel{}}
% \makeatletter
% \@addtoreset{equation}{subsection}
% \makeatother

\setlength{\parindent}{0cm}

\usepackage{hyperref}




\title{Stochastic Interest Rates 第4回}
\dept{QD部}
\author{朝倉響}
\date{2024年9月4日}

\begin{document}
\maketitle

\section*{概要}
\begin{itemize}
  \item フォワード測度を用いた金利スワップ価格付け
  \item フォワード速度を用いたオプション価格付け(BS式)
  \item キャップとフロア
\end{itemize}

\setcounter{section}{2}
\setcounter{subsection}{4}
\setcounter{equation}{11}
\subsection{FRAと金利スワップ:フォワード測度}
FRA:時刻$T$でスポットレート$L(S,T)$払い,固定$K$受け:時刻$T$で$\tau(K-L(S,T))$のペイオフ($\tau=T-S$).
フォワード測度$P_T$の下,時刻$t$でのFRAの価値$V(t)$は
\begin{align*}
  V(t) &= B(t,T)\E_{P_T}(\tau(K-L(S,T))\mid\F_t) &&\\
  &= B(t,T)(\tau K - \tau\E_{P_T}(L(S,T)\mid\F_t)) &&\text{($K$は$\F_t$--可測)} \\
  &= B(t,T)(\tau K - \tau F(t;S,T)) && \text{(式(2.11))}
\end{align*}
これが$0$となるような$K$は$K=F(t;S,T)$である.

金利スワップ:元本$1$\$で時刻$T_1,T_2,\ldots,T_n$でスポットレート$L(T_{i-1},T_i)$受け,固定金利$K$払い(ペイヤーズスワップ).

キャッシュフローの時刻$t$での価値$V(t)$は,上述のFRAの価値の和である.
\begin{align*}
  \mathbf{PS}(t) &= \sum_{i=1}^n B(t,T_i)\E_{P_{T_i}}(\tau_i(L(T_{i-1},T_i))-K\mid\F_t) \\
  &= \sum_{i=1}^n B(t,T_i)(\tau_i F(t;T_{i-1},T_i)-K) \\
  &= B(t,T_0) - B(t,T_n) -K\sum_{i=1}^n \tau_iB(t,T_i)
\end{align*}
最後の等号では,
\begin{equation*}
  \tau_i F(t;S,T) = \frac{B(t,T_{i-1})}{B(t,T_i)} - 1
\end{equation*}
を用いたが,このディスカウントファクターと現在価値計算の割引に用いるディスカウントファクターはシングルカーブの世界でのみ一致し,上の式のようにきれいな形になる.

\subsection{フォワード測度のもとでのオプション価格}
仮定:ゼロクーポンボンド$B(t,T)$が以下のSDEに従う($r(t),\Sigma(t,T)$は確定的)
\begin{equation}
  \diff B(t,T) = B(t,T)r(t)\diff t + B(t,T)\Sigma(t,T)\diff W(t) \tag{2.8}\label{eq:2.8}
\end{equation}

満期$S$,ストライク$K$の$T$満期ゼロクーポン債のコールオプション($0<S<T$). 
時刻$t<S$での価値$V(t)$は(Assumption~2.1より時刻$S$でペイオフ$(B(S,T)-K)^+$となることを念頭に)
\begin{align*}
  V(t) &= B(t)\E_Q\left(\frac{(B(S,T)-K)^+}{B(S)}\relmiddle|\F_t\right)\\
  &= B(t)\E_Q\left(\frac{B(S,T)}{B(S)}\I_{\{B(S,T)\geq K\}}\relmiddle|\F_t\right) - KB(t)\E_Q\left(\frac{1}{B(S)}\I_{\{B(S,T)\geq K\}}\relmiddle|\F_t\right)\\
\end{align*}
第1項,第2項それぞれでニューメレールの変換を行う.
\begin{itemize}
  \item 第1項:$B(t)\to B(t,T)$($Q\to P_T$).
  \item 第2項:$B(t)\to B(t,S)$($Q\to P_S$).
\end{itemize}
\begin{align}
  V(t) &= B(t,T)\E_{P_T}\left(\I_{\{B(S,T)\geq K\}}\mid\F_t\right) - KB(t,S)\E_{P_S}\left(\I_{\{B(S,T)\geq K\}}\mid\F_t\right) \notag\\
  &= B(t,T)P_T\left(B(S,T)\geq K\mid\F_t\right) - KB(t,S)P_S\left(B(S,T)\geq K\mid\F_t\right) 
\end{align}
測度$P_S,P_T$それぞれのもとでの$B(S,T)$の分布がわかればよい.

まず$P_S$について考える.
準備として,フォワードボンド価格$\FP(t;S,T)=\frac{B(t,T)}{B(t,S)}$を考える.
$f(x,y)=\frac{x}{y}$について伊藤の公式より
\begin{align*}
  f_x &= \frac{1}{y},\ f_y = -\frac{x}{y^2},\ f_{xx} = 0,\ f_{yy} = \frac{2x}{y^3},\ f_{xy} = -\frac{1}{y^2} \\
  \diff\left(\frac{x}{y}\right) &= \frac{x}{y}\left(\frac{\diff x}{x} - \frac{\diff y}{y} - \frac{\diff x\diff y}{xy} + \frac{\diff y\diff y}{y^2}\right)
\end{align*}
式\eqref{eq:2.8}より
\begin{align}
  \diff\FP(t;S,T) &= \FP(t;S,T)(r(t)\diff t+\Sigma(t,T)\diff W(t) - r(t)\diff t -\Sigma(t,S)\diff W(t)\notag\\
  &\hspace{18em} - \Sigma(t,S)\Sigma(t,T)\diff t + \Sigma(t,S)\Sigma(t,S)\diff t) \notag\\
  &= \FP(t;S,T)(\Sigma(t,T)-\Sigma(t,S))\diff W(t) - \FP(t;S,T)\Sigma(t,S)(\Sigma(t,T)-\Sigma(t,S))\diff t 
  \label{eq:2.13}
\end{align}
Proposition~2.3より,$B(t,S)$が\eqref{eq:2.8}に従うとき,フォワード測度$P_S$に関するラドンニコディム微分は
\begin{equation*}
  \frac{\diff P_S}{\diff Q} = \exp\left(\int_0^S\Sigma(u,S)\,\diff W(u) - \frac{1}{2}\int_0^S\Sigma(u,S)^2\,\diff u\right)
\end{equation*}
ギルサノフの定理より
\begin{equation}
  W^S(t) = W(t) - \int_0^t\Sigma(u,S)\,\diff u 
\end{equation}
は$P_S$に関してブラウン運動である.
式\eqref{eq:2.13}を書き直せば
\begin{equation}
  \diff\FP(t;S,T) = \FP(t;S,T)(\Sigma(t,T)-\Sigma(t,S))\diff W^S(t) 
\end{equation}
これと,$B(S,T)=\FP(S;S,T)$より
\begin{align}
  B(S,T) &= \FP(S;S,T) \notag\\
  &= \FP(t;S,T)\exp\left(\int_t^S(\Sigma(u,T)-\Sigma(u,S))\,\diff W^S(u)-\frac{1}{2}\int_t^S(\Sigma(u,T)-\Sigma(u,S))^2\,\diff u\right) \label{eq:2.16}
\end{align}
\begin{equation*}
  \ln\frac{B(S,T)}{\FP(t;S,T)} = \int_t^S(\Sigma(u,T)-\Sigma(u,S))\,\diff W^S(u) - \frac{1}{2}\int_t^S(\Sigma(u,T)-\Sigma(u,S))^2\,\diff u
\end{equation*}
ここで$\ln\frac{B(S,T)}{\FP(t;S,T)}$の分布を考える.
$\Sigma$は確定的で,$W^S$は$P_S$のもとでブラウン運動なので,右辺は$\F_t$と独立で,$P_S$のもと正規分布に従う.
伊藤積分(第1項)は平均$0$,分散\footnote{伊藤の等長性より}$\int_t^S(\Sigma(u,T)-\Sigma(u,S))^2\,\diff u$の正規分布に従うため,上式全体で$\ln\frac{B(S,T)}{\FP(t;S,T)}$は$P_S$のもと
\begin{align*}
  \text{平均} &: -\frac{1}{2}\int_t^S(\Sigma(u,T)-\Sigma(u,S))^2\,\diff u = -\frac{1}{2}v(t,S)\\
  \text{分散} &: \int_t^S(\Sigma(u,T)-\Sigma(u,S))^2\,\diff u = v(t,S)
\end{align*}
の正規分布に従う.

以上より
\begin{align}
  P_S(B(S,T)\geq K\mid\F_t) &= P_S\left(\ln\frac{B(S,T)}{\FP(t;S,T)}\geq\ln\frac{K}{\FP(t;S,T)}\relmiddle|\F_t\right) \notag\\
  &= P_S\left(\ln\frac{B(S,T)}{\FP(t;S,T)}\geq\ln\frac{K}{\FP(t;S,T)}\right) \notag\\
  &= \frac{1}{\sqrt{2\pi v(t,S)}}\int_{\ln\frac{K}{\FP(t;S,T)}}^\infty\exp\left(-\frac{\left(x+\frac{v(t,S)}{2}\right)^2}{2v(t,S)}\right)\,\diff x \notag\\
  &= \frac{1}{\sqrt{2\pi}}\int_{-d_{-}}^\infty \exp\left(-\frac{y^2}{2}\right)\,\diff y \qquad\qquad \left(y=\frac{x+\frac{v(t,S)}{2}}{\sqrt{v(t,S)}}\right) \notag\\
  &= \frac{1}{\sqrt{2\pi}}\int_{-\infty}^{d_{-}} \exp\left(-\frac{y^2}{2}\right)\,\diff y \notag\\
  &= N(d_{-}) 
\end{align}
ここで,
\begin{align}
  d_{-} &= -\frac{\ln\frac{K}{\FP(t;S,T)} + \frac{1}{2}v(t,S)}{\sqrt{v(t,S)}} 
  = -\frac{-\ln\frac{\FP(t;S,T)}{K} + \frac{1}{2}v(t,S)}{\sqrt{v(t,S)}} \notag\\
  &= \frac{\ln\frac{\FP(t;S,T)}{K} - \frac{1}{2}v(t,S)}{\sqrt{v(t,S)}} 
  = \frac{\ln\frac{B(t,T)}{KB(t,S)}-\frac{1}{2}v(t,S)}{\sqrt{v(t,S)}}\label{eq:2.18}
\end{align}

$P_T$についても考える.
Proposition~2.3とギルサノフの定理より
\begin{align*}
  \frac{\diff P_T}{\diff Q} &= \exp\left(\int_0^T\Sigma(u,T)\,\diff W(u) - \frac{1}{2}\int_0^T\Sigma(u,T)^2\,\diff u\right) \\
  W^T(t) &= W(t) - \int_0^t\Sigma(u,T)\,\diff u
\end{align*}
で,$W^T(t)$は$P_T$のもとでブラウン運動.
$W^S(t)$との関係
\begin{equation*}
  W^S(t) = W(t) - \int_0^t\Sigma(u,S)\,\diff u = W^T(t) + \int_0^t(\Sigma(u,T)-\Sigma(u,S))\,\diff u
\end{equation*}
を\eqref{eq:2.16}に用いれば
\begin{equation*}
  B(S,T) = \FP(t;S,T)\exp\left(\int_t^S(\Sigma(u,T)-\Sigma(u,S))\,\diff W^T(u) + \frac{1}{2}\int_t^S(\Sigma(u,T)-\Sigma(u,S))^2\,\diff u\right)
\end{equation*}
フォワード測度$P_T$のもとでの$\ln\frac{B(S,T)}{\FP(t;S,T)}$は平均$\frac{1}{2}v(t,S)$,分散$v(t,S)$の正規分布に従う.
先と同様にして
\begin{align}
  P_T(B(S,T)\geq K\mid\F_t) &= P_T\left(\ln\frac{B(S,T)}{\FP(t;S,T)}\geq\ln\frac{K}{\FP(t;S,T)}\right) = N(d_{+})\label{eq:2.19}\\
  d_{+} &= \frac{\ln\frac{\FP(t;S,T)}{K}+\frac{1}{2}v(t,S)}{\sqrt{v(t,S)}} = \frac{\ln\frac{B(t,T)}{KB(t,S)}+\frac{1}{2}v(t,S)}{\sqrt{v(t,S)}}\label{eq:2.20}
\end{align}

\setcounter{thm}{3}
\begin{thm}
  ゼロクーポン債$B(t,T)$が式\eqref{eq:2.8}に従うとき,満期$S$,ストライク$K$のゼロクーポン債$B(t,T)$のコールオプションの価値$\BC(t;S,T,K)$は
  \begin{equation}
    \BC(t) = B(t,T)N(d_{+}) - KB(t,S)N(d_{-}) \label{eq:2.21}
  \end{equation}
  であり,$d_{\pm}$は式\eqref{eq:2.18},\eqref{eq:2.20}で与えられる.
\end{thm}

\subsubsection*{債権オプションのプット・コール・パリティ}
満期$S$で行使価格$K$のコールの買いとプットの売りのポートフォリオを考える.
時刻$S$での価値は
\begin{equation*}
  (B(S,T)-K)^+ - (K-B(S,T))^+ = B(S,T) - K
\end{equation*}
となり,これは満期$T$のゼロクーポン債の1単位買いと満期$S$のゼロクーポン債の$K$単位買いのポートフォリオと等価である.
無裁定のもと,時刻$t$での価値も等価となり,以下の関係(債権オプションの\textbf{プットコールパリティ})が成り立つ
\begin{equation*}
  \BC(t;S,T,K) - \BP(t;S,T,K) = B(t,T) - KB(t,S)
\end{equation*}
ここで,$\BP(t;S,T,K)$は満期$S$で行使価格$K$の$T$-bondプットオプションの$t$時点での価値.

プットコールパリティを使えば,$\BC(t;S,T,K)$の式\eqref{eq:2.21}から$\BP(t;S,T,K)$を求めることができる.
\begin{align*}
  \BP(t;S,T,K) &= \BC(t;S,T,K) - B(t,T) + KB(t,S) \\
  &= B(t,T)N(d_{+}) - KB(t,S)N(d_{-}) - B(t,T) + KB(t,S) \\
  &= KB(t,S)(1-N(d_{-})) - B(t,T)(1-N(d_{+}))
\end{align*}
$N(\cdot)$の定義(標準正規累積分布関数)より
\begin{align*}
  1-N(x) &= 1-\frac{1}{\sqrt{2\pi}}\int_{-\infty}^x \exp\left(-\frac{y^2}{2}\right)\,\diff y\\
  &= \frac{1}{\sqrt{2\pi}}\int_{x}^\infty \exp\left(-\frac{y^2}{2}\right)\,\diff y\\
  &= \frac{1}{\sqrt{2\pi}}\int_{-\infty}^{-x} \exp\left(-\frac{y^2}{2}\right)\,\diff y = N(-x)
\end{align*}
が成り立つため
\begin{equation}
  \BP(t;S,T,K) = KB(t,S)N(-d_{-}) - B(t,T)N(-d_{+}) \label{eq:2.22}
\end{equation}

\subsection{キャップとフロア}
金利キャップ(もしくはフロア):時刻$T_i\,(i=1,\ldots,n)$で$\tau_i(L(T_{i-1},T_i)-K)^{+}$(もしくは$\tau_i(K-L(T_{i-1},T_i))^{+}$)を受け取る($\tau_i=T_i-T_{i-1}$).
各時刻$T_i$でのペイオフを\textbf{キャップレット}(もしくは\textbf{フロアレット})と呼ぶ.
$T_i$のキャップレットの$T_{i-1}$での価値は
\begin{equation*}
  \tau_iB(T_{i-1},T_i)(L(T_{i-1},T_i)-K)^{+}
\end{equation*}
であるが,シングルカーブの世界では,LIBORレート$L(T_{i-1},T_i)$とディスカウントファクターの関係
\begin{align}
  B(T_{i-1},T_i) &= \frac{1}{1+\tau_iL(T_{i-1},T_i)} \notag\\
  L(T_{i-1},T_i) &= \frac{1-B(T_{i-1},T_i)}{\tau_iB(T_{i-1},T_i)} \notag
\end{align}
を用いて,
\begin{align*}
  \tau_iB(T_{i-1},T_i)(L(T_{i-1},T_i)-K)^{+} &= \tau_iB(T_{i-1},T_i)\left(\frac{1-B(T_{i-1},T_i)}{\tau_iB(T_{i-1},T_i)}-K\right)^{+} \\
  &= \left(1-B(T_{i-1},T_i)-\tau_iKB(T_{i-1},T_i)\right)^{+}\\
  &= (1+\tau_iK)\left(\frac{1}{1+\tau_iK}-B(T_{i-1},T_i)\right)^{+}
\end{align*}
と変形される.
これは,$T_i$満期のゼロクーポン債に基づく,ストライク$\frac{1}{1+\tau_iK}$で満期$T_{i-1}$のプットオプションを$1+\tau_iK$単位持つことに等しい.
よって,キャップの価値は,プットオプションの価値の和で表せることになる(フロアも同様).
\begin{align*}
  \mathbf{Cap}(t) &= \sum_{i=1}^n (1+\tau_iK)\BP\left(t;T_{i-1},T_i,\frac{1}{1+\tau_iK}\right) \\
  \mathbf{Flr}(t) &= \sum_{i=1}^n (1+\tau_iK)\BC\left(t;T_{i-1},T_i,\frac{1}{1+\tau_iK}\right)
\end{align*}

\subsubsection*{キャップとフロアのプット・コール・パリティ}
キャップ1単位の買いとフロア1単位の売りのポートフォリオを考える.
時刻$T_i\,(i=1,2,\ldots,n)$でのペイオフは$\tau_i(L(T_{i-1},T_i)-K)$であり,ペイヤースワップの時刻$T_i$でのペイオフと等価である:
\begin{align*}
  \mathbf{Cap}(t) - \mathbf{Flr}(t) &= \mathbf{PS}(t) 
\end{align*}

\end{document}